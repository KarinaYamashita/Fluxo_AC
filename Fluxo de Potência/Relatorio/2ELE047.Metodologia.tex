\section{METODOLOGIA EXPERIMENTAL}
Como dito anteriormente, esta atividade teve como objetivo o desenvolvimento de algor�timos capaz de resolver o problema de fluxo de carga. Para test�-los, foi utilizado um problema composto por 14 barras.

As barras do sistema em quest�o possuem as caracter�sticas apresentadas abaixo. As barras est�o enumeradas e classificadas quanto ao seu tipo, sendo o tipo 0 barras PQ, o tipo 2 barras PV e tipo 3 barras $V\theta$. As barras que n�o possuem tens�o controladas tomar�o valores unit�rios (em pu) e os �ngulos das tens�o ser�o considerados nulos. Al�m disso, as pot�ncias ser�o transformadas em pu, sendo a unidade base $100\,[kVA]$
\begin{table}[H]
	\centering
	\scriptsize
	\begin{tabular}{c|c|c|c|c|c|c}\hline
	\multirow{2}{*}{\textbf{Barra}}   & \multirow{2}{*}{\textbf{Tipo}} & \textbf{Tens�o} & \multicolumn{2}{ |c| }{\textbf{Pot�ncia Gerada [kVA]}} & \multicolumn{2}{ |c }{\textbf{Pot�ncia demandada [kVA]}} \\\cline{4-7}
     & & \textbf{Controlada [pu]} & \textbf{Ativa} & \textbf{Reativa} & \textbf{Ativa} & \textbf{Reativa}\\\hline
	%\textbf{Barra} & \textbf{Tipo} & \textbf{Tens�o} & \textbf{Pot�ncia gerada}
	$1$ & $3$ & $1.060E+00$ & $2.324E+02$ & $-1.690E+01$ & $0.000E+00$ & $0.000E+00$   \\\hline
	$2$ & $2$ & $1.045E+00$ & $4.000E+01$ & $4.240E+01$ & $2.170E+01$ & $1.270E+01$   \\\hline
	$3$ & $2$ & $1.010E+00$ & $0.000E+00$ & $2.340E+01$ & $9.420E+01$ & $1.900E+01$   \\\hline
	$4$ & $0$ & $0.000E+00$ & $0.000E+00$ & $0.000E+00$ & $4.780E+01$ & $-3.900E+00$   \\\hline
	$5$ & $0$ & $0.000E+00$ & $0.000E+00$ & $0.000E+00$ & $7.600E+00$ & $1.600E+00$   \\\hline
	$6$ & $2$ & $1.070E+00$ & $0.000E+00$ & $1.220E+01$ & $1.120E+01$ & $7.500E+00$   \\\hline
	$7$ & $0$ & $0.000E+00$ & $0.000E+00$ & $0.000E+00$ & $0.000E+00$ & $0.000E+00$   \\\hline
	$8$ & $2$ & $1.090E+00$ & $0.000E+00$ & $1.740E+01$ & $0.000E+00$ & $0.000E+00$   \\\hline
	$9$ & $0$ & $0.000E+00$ & $0.000E+00$ & $0.000E+00$ & $2.950E+01$ & $1.660E+01$   \\\hline
	$10$ & $0$ & $0.000E+00$ & $0.000E+00$ & $0.000E+00$ & $9.000E+00$ & $5.800E+00$   \\\hline
	$11$ & $0$ & $0.000E+00$ & $0.000E+00$ & $0.000E+00$ & $3.500E+00$ & $1.800E+00$   \\\hline
	$12$ & $0$ & $0.000E+00$ & $0.000E+00$ & $0.000E+00$ & $6.100E+00$ & $1.600E+00$   \\\hline
	$13$ & $0$ & $0.000E+00$ & $0.000E+00$ & $0.000E+00$ & $1.350E+01$ & $5.800E+00$   \\\hline
	$14$ & $0$ & $0.000E+00$ & $0.000E+00$ & $0.000E+00$ & $1.490E+01$ & $5.000E+00$   \\\hline
	\end{tabular}
	\caption{Caracter�sticas das barras}
\end{table}


Os par�metros da linha, j� expressos em pu, s�o mostrados na tabela subsequente:
\begin{table}[H]
	\centering
	\scriptsize
	\begin{tabular}{c|c|c|c|c}\hline
	\textbf{De} & \textbf{Para} & $\mathbf{r_{ij}}$ & $\mathbf{x_{ij}}$ & $\mathbf{Bsh_{ij}}$\\\hline
	$1$  & $2$  & $1.938E-02$ & $5.917E-02$ & $5.280E-02$   \\\hline
	$1$  & $5$  & $5.403E-02$ & $2.230E-01$ & $4.920E-02$   \\\hline
	$2$  & $3$  & $4.699E-02$ & $1.980E-01$ & $4.380E-02$   \\\hline
	$2$  & $4$  & $5.811E-02$ & $1.763E-01$ & $3.400E-02$   \\\hline
	$2$  & $5$  & $5.695E-02$ & $1.739E-01$ & $3.460E-02$   \\\hline
	$3$  & $4$  & $6.701E-02$ & $1.710E-01$ & $1.280E-02$   \\\hline
	$4$  & $5$  & $1.335E-02$ & $4.211E-02$ & $0.000E+00$   \\\hline
	$4$  & $7$  & $0.000E+00$ & $2.091E-01$ & $0.000E+00$   \\\hline
	$4$  & $9$  & $0.000E+00$ & $5.562E-01$ & $0.000E+00$   \\\hline
	$5$  & $6$  & $0.000E+00$ & $2.520E-01$ & $0.000E+00$   \\\hline
	$6$  & $11$ & $9.498E-02$ & $1.989E-01$ & $0.000E+00$   \\\hline
	$6$  & $12$ & $1.229E-01$ & $2.558E-01$ & $0.000E+00$   \\\hline
	$6$  & $13$ & $6.615E-02$ & $1.303E-01$ & $0.000E+00$   \\\hline
	$7$  & $8$  & $0.000E+00$ & $1.762E-01$ & $0.000E+00$   \\\hline
	$7$  & $9$  & $0.000E+00$ & $1.100E-01$ & $0.000E+00$   \\\hline
	$9$  & $10$ & $3.181E-02$ & $8.450E-02$ & $0.000E+00$   \\\hline
	$9$  & $14$ & $1.271E-01$ & $2.704E-01$ & $0.000E+00$   \\\hline
	$10$ & $11$ & $8.205E-02$ & $1.921E-01$ & $0.000E+00$   \\\hline
	$12$ & $13$ & $2.209E-01$ & $1.999E-01$ & $0.000E+00$   \\\hline
	$13$ & $14$ & $1.709E-01$ & $3.480E-01$ & $0.000E+00$   \\\hline
	\end{tabular}
	\caption{Caracter�sticas das linhas}
\end{table}
